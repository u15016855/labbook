\chapter[2019 April]{April 2019}

\section[2019/04/01]{Someday, 1 April 2019}
\label{sec:20190401}

\lipsum
\begin{align}
  \m{QX} \m{a} & = \m{Qy} \notag \\
  \therefore \quad (\m{QX})^T \m{QX} \m{a} & = (\m{QX})^T \m{Qy} \notag \\
  \therefore \quad \m{a} & = \left( (\m{QX})^T \m{QX} \right)^{-1} (\m{QX})^T \m{Qy} \notag \\
        & = \left( \mt{X} \mt{Q} \m{QX} \right)^{-1} \mt{X} \mt{Q} \m{Qy} \notag \\
        & = \left( \mt{X} \m{W} \m{X} \right)^{-1} \mt{X} \m{W} \m{y}
          \label{eqn:weightedleastsquares}
\end{align}
A subsection a subsection a subsection a subsection a subsection a
subsection a subsection a subsection a subsection a subsection a
subsection a subsection a subsection a subsection a subsection.
$\mathcal O$

\pendsign

\section[2019/04/02]{Someday, 2 April 2019}

\begin{lstlisting}[language=Python,title={Monte Carlo approximation of $\pi$}]
import random
import math

count_inside = 0

for count in range(0, 10000):
    d = math.hypot(random.random(), random.random())
    if d < 1:
        count_inside += 1
count += 1
print("%0.8f" % (4.0 * count_inside / count))
\end{lstlisting}

